\documentclass{article}

\usepackage[spanish]{babel}

\usepackage[letterpaper,top=2cm,bottom=2cm,left=3cm,right=3cm,marginparwidth=1.75cm]{geometry}

\usepackage{graphicx}
\usepackage[colorlinks=true, allcolors=blue]{hyperref}
\usepackage[strict]{changepage}
\usepackage{amsmath,amsthm,amssymb,amsfonts, color, comment, graphicx, environ}
\usepackage{xcolor}
\usepackage{listings}
\usepackage{tcolorbox}
\usepackage{mdframed}
\usepackage{tikz}
\usepackage{tkz-fct}
\usetikzlibrary{calc,arrows,babel,shapes,spy,positioning,snakes}
\usepackage{mathtools}
\usepackage{physics}
\usepackage{calligra}
\usepackage{csquotes}
\usepackage{tensor}
\usepackage[thicklines]{cancel}
\usepackage{pstricks}
\usepackage{fancyhdr}
\usepackage{multicol}
\usepackage{float}
\usepackage[shortlabels]{enumitem}
\usepackage{indentfirst}
\usepackage{hyperref}
%
\hypersetup{
	colorlinks=true,
	linkcolor=blue,
	filecolor=magenta,      
	urlcolor=blue,
}
\usepackage[T1]{fontenc}
\usepackage{titlesec}

%%%%%%%%%%%%%%%%%%%% Colores %%%%%%%%%%%%%%%%%%%%

%%%%% COLORES PASTEL %%%%%
\definecolor{moradopastel}{RGB}{213,201,255}
\definecolor{rosapastel}{RGB}{254,202,202}
\definecolor{azulpastel}{RGB}{217,238,251}
\definecolor{verdepastel}{RGB}{204,252,220}
\definecolor{rojopastel}{RGB}{255,109,109}
%%%%% COLORES PASTEL %%%%%

%%%%% GRISES %%%%%
\definecolor{grisclaro}{RGB}{247, 249, 249}
\definecolor{grisclaro2}{RGB}{240, 243, 244}
\definecolor{gris}{RGB}{191, 201, 202}
%%%%% GRISES %%%%%

%%%%% HTML page %%%%%
\definecolor{IndianRed}{RGB}{205, 92, 92}
\definecolor{LightCoral}{RGB}{240, 128, 128}
\definecolor{Salmon}{RGB}{250, 128, 114}
\definecolor{DarkSalmon}{RGB}{233, 150, 122}
\definecolor{LightSalmon}{RGB}{255, 160, 122}
%%%%% HTML page %%%%%

%%%%% MIS COLORES %%%%%
\definecolor{Rojop}{RGB}{203, 67, 53}
\definecolor{Azulp}{RGB}{41, 128, 185}
\definecolor{Verdep}{RGB}{20, 120, 100}
\definecolor{Amarillop}{RGB}{255, 187,0}
\definecolor{Naranjap}{RGB}{230, 126, 34}
\definecolor{Moradop}{RGB}{108, 99, 255}
\definecolor{Blancop}{RGB}{254, 254, 254}
\definecolor{Negrop}{RGB}{0,0,0}
\definecolor{Cyanp}{RGB}{118, 215, 196}
\definecolor{Olivop}{RGB}{20, 90, 50}
%%%%% MIS COLORES %%%%%

%%%%%%%%%% CODING COLORS %%%%%%%%%%
\definecolor{backgroundcolorgray}{RGB}{61,61,61}
\definecolor{codeYellow}{RGB}{255,216,61}
\definecolor{codeWhite}{RGB}{235,235,235}
\definecolor{codePink}{RGB}{255,0,235}
\definecolor{codeCyan}{RGB}{0,255,255}
\definecolor{codeRed}{RGB}{255,0,55}
%%%%%%%%%% CODING COLORS %%%%%%%%%%

%%%%%%%%%% MANIM COLORS %%%%%%%%%%%
\definecolor{REDmanim}{RGB}{252, 98, 85}
\definecolor{GREENmanim}{RGB}{131, 193, 103}
\definecolor{BLUEmanim}{RGB}{88, 196, 221}
\definecolor{ORANGEmanim}{RGB}{255, 134, 47}
\definecolor{PURPLEmanim}{RGB}{154, 114, 172}
%%%%%%%%%% MANIM COLORS %%%%%%%%%%%

%%%%%%%%%%%%%%%%%%%% Colores %%%%%%%%%%%%%%%%%%%%

%%%%%%%%%%%%%%%%%%%% Comandos renovados %%%%%%%%%%%%%%%%%%%%

\renewcommand{\qed}{\quad\qedsymbol}
\renewcommand{\qed}{\quad\qedsymbol}
\renewcommand{\theenumi}{\alph{enumi})}

%%%%%%%%%%%%%%%%%%%% Comandos renovados %%%%%%%%%%%%%%%%%%%%

%%%%%%%%%%%%%%%%%%%% Nuevos comandos %%%%%%%%%%%%%%%%%%%%
\makeatletter
\providecommand*{\cupdot}{%
	\mathbin{%
		\mathpalette\@cupdot{}%
	}%
}
\newcommand*{\@cupdot}[2]{%
	\ooalign{%
		$\m@th#1\cup$\cr
		\hidewidth$\m@th#1\cdot$\hidewidth
	}%
}
\makeatother
\newcommand{\scriptr}{\mathcalligra{r}\,}
\newcommand{\boldscriptr}{\pmb{\mathcalligra{r}}\,}
\newcommand{\ie}{\emph{i.e.}} %id est
\newcommand{\eg}{\emph{e.g.}} %exempli gratia
\newcommand{\rtd}[1]{\ensuremath{\left\lfloor #1 \right\rfloor}}
\newcommand{\dirac}[1]{\ensuremath{\delta \left( #1 \right)}}
\newcommand{\diract}[1]{\ensuremath{\delta^3 \left( #1 \right)}}
\newcommand{\e}{\ensuremath{\epsilon_0}}
\newcommand{\m}{\ensuremath{\mu_0}}
\newcommand{\V}{\ensuremath{\mathcal{V}}}
\newcommand{\prnt}[1]{\ensuremath{\left(#1\right)}} %parentheses
\newcommand{\colch}[1]{\ensuremath{\left[#1\right]}} %square brackets
\newcommand{\chave}[1]{\ensuremath{\left\{#1\right\}}}  %curly brackets
\newcommand{\hlight}[1]{\colorbox{violet!50}{#1}}
\newcommand{\hlighta}[1]{\colorbox{red!50}{#1}}

\newcommand{\ai}{\'{\i}}

\newcommand{\R}{\mathbb{R} }
\newcommand{\N}{\mathbb{N} }
\newcommand{\Q}{\mathbb{Q} }
\newcommand{\Z}{\mathbb{Z} }
\newcommand{\I}{\mathbb{I} }
\newcommand{\subSpace}{\textbf{W}}
\newcommand{\spn}[1]{\operatorname{\texttt{span}}\left(#1\right)}
\newcommand{\gen}[1]{\operatorname{\texttt{gen}}\left(#1\right)}
\newcommand{\field}{\mathbb{F} }
\newcommand{\vectorSpace}{\textbf{V} }
\newcommand{\Zx}[1]{\mathbb{R}_{#1} }
\renewcommand{\qed}{\quad\qedsymbol}
\newcommand{\mcd}[2]{$\operatorname{mcd}\left(#1,#2\right)$}
\newcommand{\tituloS}[1]{\textbf{#1}\newline}
\newcommand{\coma}{,\:}
\newcommand{\y}{\:y\:}
%\newcommand{\op}{$\:o\:$}

%%%%%%%%%%%%%%%%%%%% Nuevos comandos %%%%%%%%%%%%%%%%%%%%


%%%%%%%%%%%%%%%%%%%% Entornos %%%%%%%%%%%%%%%%%%%%

%%%%%%%%%%%%%%%% Problema %%%%%%%%%%%%%%%
\newcounter{problem}[section]\setcounter{problem}{0}
\renewcommand{\theproblem}{\textcolor{Blancop}{\thesection.\arabic{problem}}}
\newenvironment{problem}[2][]{%
	\refstepcounter{problem}%
	\ifstrempty{#2}%
	{\mdfsetup{%
			frametitle={%
				\tikz[baseline=(current bounding box.east),outer sep=0pt]
				\node[anchor=east,rectangle,fill=IndianRed]
				{\strut \textcolor{Blancop}{Problema}~\theproblem};}}
	}%
	{\mdfsetup{%
			frametitle={%
				\tikz[baseline=(current bounding box.east),outer sep=0pt]
				%\node[anchor=east,rectangle,fill=IndianRed]
				{\strut \textcolor{Blancop}{Problema}~\theproblem:~#2};}}%
	}%
	\mdfsetup{innertopmargin=10pt,linecolor=IndianRed,%
		linewidth=3pt,topline=true,bottomline=false,leftline=false,rightline=false,%
		frametitleaboveskip=\dimexpr-\ht\strutbox\relax
	}
	\begin{mdframed}[]\relax%
		\label{#2}}{\end{mdframed}}
%%%%%%%%%%%%%%%% Problema %%%%%%%%%%%%%%%

%%%%%%%%%%%%%%%% Definición %%%%%%%%%%%%%%%
\newcounter{definition}[section]\setcounter{definition}{0}
\renewcommand{\thedefinition}{\textcolor{Blancop}{\thesection.\arabic{definition}}}
\newenvironment{definition}[2][]{%
	\refstepcounter{definition}%
	\ifstrempty{#2}%
	{\mdfsetup{%
			frametitle={%
				\tikz[baseline=(current bounding box.east),outer sep=0pt]
				\node[anchor=east,rectangle,fill=Cyanp]
				{\strut \textcolor{Blancop}{Definición~\thedefinition}};}}
	}%
	{\mdfsetup{%
			frametitle={%
				\tikz[baseline=(current bounding box.east),outer sep=0pt]
				\node[anchor=east,rectangle,fill=Cyanp]
				{\strut \textcolor{Blancop}{Definición~\thedefinition:~#2}};}}%
	}%
	\mdfsetup{innertopmargin=10pt,linecolor=Cyanp,%
		linewidth=3pt,topline=true,bottomline=false,leftline=false,rightline=false,%
		frametitleaboveskip=\dimexpr-\ht\strutbox\relax
	}
	\begin{mdframed}[]\relax%
		\label{#2}}{\end{mdframed}}
%%%%%%%%%%%%%%%% Definición %%%%%%%%%%%%%%%

%%%%%%%%%%%%%%%% Teorema %%%%%%%%%%%%%%%
\newcounter{theorem}[section]\setcounter{theorem}{0}
\renewcommand{\thetheorem}{\textcolor{Blancop}{\thesection.\arabic{theorem}}}
\newenvironment{theorem}[2][]{%
	\refstepcounter{theorem}%
	\ifstrempty{#2}%
	{\mdfsetup{%
			frametitle={%
				\tikz[baseline=(current bounding box.east),outer sep=0pt]
				\node[anchor=east,rectangle,fill=Verdep]
				{\strut \textcolor{Blancop}{Teorema~\thetheorem}};}}
	}%
	{\mdfsetup{%
			frametitle={%
				\tikz[baseline=(current bounding box.east),outer sep=0pt]
				\node[anchor=east,rectangle,fill=Verdep]
				{\strut \textcolor{Blancop}{Teorema~\thetheorem:~#2}};}}%
	}%
	\mdfsetup{innertopmargin=10pt,linecolor=Verdep,%
		linewidth=3pt,topline=true,bottomline=false,leftline=false,rightline=false,%
		frametitleaboveskip=\dimexpr-\ht\strutbox\relax
	}
	\begin{mdframed}[]\relax%
		\label{#2}}{\end{mdframed}}
%%%%%%%%%%%%%%%% Teorema %%%%%%%%%%%%%%%

%%%%%%%%%%%%%%%% Corolario %%%%%%%%%%%%%%%
\newcounter{corollary}[section]\setcounter{corollary}{0}
\renewcommand{\thecorollary}{\textcolor{Blancop}{\thesection.\arabic{corollary}}}
\newenvironment{corollary}[2][]{%
	\refstepcounter{corollary}%
	\ifstrempty{#2}%
	{\mdfsetup{%
			frametitle={%
				\tikz[baseline=(current bounding box.east),outer sep=0pt]
				\node[anchor=east,rectangle,fill=Amarillop]
				{\strut \textcolor{Blancop}{Corolario~\thecorollary}};}}
	}%
	{\mdfsetup{%
			frametitle={%
				\tikz[baseline=(current bounding box.east),outer sep=0pt]
				\node[anchor=east,rectangle,fill=Amarillop]
				{\strut \textcolor{Blancop}{Corolario~\thecorollary:~#2}};}}%
	}%
	\mdfsetup{innertopmargin=10pt,linecolor=Amarillop,%
		linewidth=3pt,topline=true,bottomline=false,leftline=false,rightline=false,%
		frametitleaboveskip=\dimexpr-\ht\strutbox\relax
	}
	\begin{mdframed}[]\relax%
		\label{#2}}{\end{mdframed}}
%%%%%%%%%%%%%%%% Corolario %%%%%%%%%%%%%%%

%%%%%%%%%%%%%%%% Axioma %%%%%%%%%%%%%%%
\newcounter{axiom}[section]\setcounter{axiom}{0}
\renewcommand{\theaxiom}{\textcolor{Blancop}{\thesection.\arabic{axiom}}}
\newenvironment{axiom}[2][]{%
	\refstepcounter{axiom}%
	\ifstrempty{#2}%
	{\mdfsetup{%
			frametitle={%
				\tikz[baseline=(current bounding box.east),outer sep=0pt]
				\node[anchor=east,rectangle,fill=Rojop]
				{\strut \textcolor{Blancop}{Axioma}~\theaxiom};}}
	}%
	{\mdfsetup{%
			frametitle={%
				\tikz[baseline=(current bounding box.east),outer sep=0pt]
				%\node[anchor=east,rectangle,fill=Rojop]
				{\strut \textcolor{Blancop}{Axioma}~\theaxiom:~#2};}}%
	}%
	\mdfsetup{innertopmargin=10pt,linecolor=Rojop,%
		linewidth=3pt,topline=true,bottomline=false,leftline=false,rightline=false,%
		frametitleaboveskip=\dimexpr-\ht\strutbox\relax
	}
	\begin{mdframed}[]\relax%
		\label{#2}}{\end{mdframed}}
%%%%%%%%%%%%%%%% Axioma %%%%%%%%%%%%%%%

%%%%%%%%%%%%%%%% Propiedad %%%%%%%%%%%%%%%
\newcounter{property}[section]\setcounter{property}{0}
\renewcommand{\theproperty}{\textcolor{Blancop}{\thesection.\arabic{property}}}
\newenvironment{property}[2][]{%
	\refstepcounter{property}%
	\ifstrempty{#2}%
	{\mdfsetup{%
			frametitle={%
				\tikz[baseline=(current bounding box.east),outer sep=0pt]
				\node[anchor=east,rectangle,fill=Moradop]
				{\strut \textcolor{Blancop}{Propiedad~\theproperty}};}}
	}%
	{\mdfsetup{%
			frametitle={%
				\tikz[baseline=(current bounding box.east),outer sep=0pt]
				\node[anchor=east,rectangle,fill=Moradop]
				{\strut \textcolor{Blancop}{Propiedad~\theproperty:~#2}};}}%
	}%
	\mdfsetup{innertopmargin=10pt,linecolor=Moradop,%
		linewidth=3pt,topline=true,bottomline=false,leftline=false,rightline=false,%
		frametitleaboveskip=\dimexpr-\ht\strutbox\relax
	}
	\begin{mdframed}[]\relax%
		\label{#2}}{\end{mdframed}}
%%%%%%%%%%%%%%%% Postulado %%%%%%%%%%%%%%%

%%%%%%%%%%%%%%%% Lema %%%%%%%%%%%%%%%
\newcounter{motto}[section]\setcounter{motto}{0}
\renewcommand{\themotto}{\textcolor{Blancop}{\thesection.\arabic{motto}}}
\newenvironment{motto}[2][]{%
	\refstepcounter{motto}%
	\ifstrempty{#2}%
	{\mdfsetup{%
			frametitle={%
				\tikz[baseline=(current bounding box.east),outer sep=0pt]
				\node[anchor=east,rectangle,fill=Naranjap]
				{\strut \textcolor{Blancop}{Lema}~\themotto};}}
	}%
	{\mdfsetup{%
			frametitle={%
				\tikz[baseline=(current bounding box.east),outer sep=0pt]
				%\node[anchor=east,rectangle,fill=Naranjap]
				{\strut \textcolor{Blancop}{Lema}~\themotto:~#2};}}%
	}%
	\mdfsetup{innertopmargin=10pt,linecolor=Naranjap,%
		linewidth=3pt,topline=true,bottomline=false,leftline=false,rightline=false,%
		frametitleaboveskip=\dimexpr-\ht\strutbox\relax
	}
	\begin{mdframed}[]\relax%
		\label{#2}}{\end{mdframed}}
%%%%%%%%%%%%%%%% Lema %%%%%%%%%%%%%%%

%%%%%%%%%%%%%%%% Solución %%%%%%%%%%%%%%%
\newenvironment{solution}[2][Soluci\'on]{\textbf{#1 #2} \\}
%%%%%%%%%%%%%%%% Solución %%%%%%%%%%%%%%%

%%%%%%%%%%%%%%%% Demostración %%%%%%%%%%%%%%%
\newenvironment{demostration}[2][Demostraci\'on]{\textbf{#1 #2} \\}
%%%%%%%%%%%%%%%% Demostración %%%%%%%%%%%%%%%

%%%%%%%%%%%%%%%% Ejemplo %%%%%%%%%%%%%%%
\newenvironment{example}[2][Ejemplo]{\textbf{#1 #2} \\}
%%%%%%%%%%%%%%%% Ejemplo %%%%%%%%%%%%%%%

%%%%%%%%%%%%%%%%%%%% Entornos %%%%%%%%%%%%%%%%%%%%

